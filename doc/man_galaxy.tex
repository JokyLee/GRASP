% /* GRASP: Copyright 1997,1998  Bruce Allen */
% $Id: man_galaxy.tex,v 1.5 1999/07/11 21:22:11 ballen Exp $
\section{Galactic Modelling}
\label{s:galaxy}
\setcounter{equation}0
Large parts of GRASP were developed using data from the Caltech
40-meter prototype detector.  This detector is typically sensitive
to distances of tens of kiloparsecs, roughly the length scale of our
galaxy.  It was therefore useful to develop a model of the galactic
distributions of expected sources, for example, of the galactic
distribution
of binary star systems.  This section contains a package of routines
which can be used to simulate the expected distribution of binary
inspiral
sources in our galaxy.  Because of the earth's motion about its axis
and about the sun, the distribution of expected sources (direction,
distance, orientation) changes as a function of time, with roughly a
24-hour period.  This section constains routines which model the
distribution of these parameters as a function of the earth's motion.
\clearpage

\subsection{Function: \texttt{local\_sidereal\_time()}}
\label{ss:lst}

\begin{verbatim}
float local_sidereal_time(time_t time, float longitude)
\end{verbatim}

Returns the local sidereal time, in decimal hours, for a given calendar time
and detector longitude.  The arguments are:
\begin{description}
\item{\texttt{time}}: Input.  The time as an integer number of seconds since
  0h 1~January 1970.  This number is returned by the time routines in
  \texttt{<time.h>}.
\item{\texttt{longitude}}: Input.  The longitude of the detector in degrees
  West.
\end{description}

The local sidereal time is calculated as follows.  Let JD be the Julian date
of 0h on the desired calendar day.  (This is computed using the Numerical
Recipes routine \texttt{julday()}, but a value of 0.5 must be subtracted from
this routine to give the JD at 0h rather than at 12h.)  The
Universal Time, UT, is computed using the \texttt{gmtime()} function.
The Greenwich Sidereal Time, GST, is
\begin{displaymath}
  \mbox{GST} = T_0 + 1.002\,737\,909\times\mbox{UT}
\end{displaymath}
(modulo 24 hours) where
\begin{displaymath}
  T_0 = 6.697\,374\,558 + 2\,400.051\,336\times T + 0.000\,025\,862\times T^2
\end{displaymath}
and
\begin{displaymath}
  T = \frac{\mbox{JD}-2\,451\,545}{36\,525}.
\end{displaymath}
The \emph{local} sidereal time is obtained by subtracting the longitude of the
detector expressed as decimal hours West of Greenwich.

\begin{description}
\item{Author:} Jolien Creighton, jolien@tapir.caltech.edu
\item{Comments:}  This routine is adapted from the method given in:
  Peter Duffet-Smith \emph{Practical Astronomy with Your Calculator,}
  3rd edition,
  (Cambridge University Press, 1988).
\end{description}

\clearpage
\subsection{Example: \texttt{caltech\_lst} program}
\label{ss:caltech_lst}

This program is an example of how to use the function
\texttt{local\_sidereal\_time()} to compute the local sidereal time
at the Caltech 40-meter laboratory.  The program determines the
present sidereal time at Caltech if it receives no arguments.  Instead,
the program can receive a single long integer specifying the number of
seconds since 0h UTC 1 Jan 1970, and it returns the Caltech sidereal
time at that time.  Also returned are the number of seconds since
0h UTC 1 Jan 1970, the local and coordinated universal times, and the
Greenwitch sidereal time.  Some examples:
\begin{enumerate}
\item \verb|caltech_lst|
\begin{verbatim}
881889773 seconds since 0h UTC 1 Jan 1970
17:22:53 PST Thu 11 Dec 1997
01:22:53 UTC Fri 12 Dec 1997
22:53:30 (22.891788 h) Local Sidereal Time
06:46:02 (06.767323 h) Greenwitch Sidereal Time
\end{verbatim}
(guess when I wrote this example!)
\item \verb|caltech_lst 0|
\begin{verbatim}
0 seconds since 0h UTC 1 Jan 1970
16:00:00 PST Wed 31 Dec 1969
00:00:00 UTC Thu 01 Jan 1970
22:48:23 (22.806442 h) Local Sidereal Time
06:40:55 (06.681976 h) Greenwitch Sidereal Time
\end{verbatim}
\end{enumerate}

The program can be simply modified to correspond to \emph{your} local
sidereal time by changing the \verb|#define LONGITUDE| command to your
local longitude (in degrees West of Greenwitch).

\begin{description}
\item[author:] Jolien Creighton, jolien@tapir.caltech.edu
\end{description}

\clearpage
\lgrindfile{Includes/caltech_lst.tex}

\clearpage
\subsection{Function: \texttt{galactic\_to\_equatorial()}}
\label{ss:galactic_to_equatorial}

\begin{verbatim}
void galactic_to_equatorial(float l, float b, float *alpha, float *delta)
\end{verbatim}

This routine converts the coordinates of an object from the Galactic
system---Galactic longitude~$\ell$ and latitude~$b$---to the equatorial
system---right ascension~$\alpha$ and declination~$\delta$.  The arguments are:
\begin{description}
\item{\texttt{l}}: Input.  The Galactic longitude~$\ell$ (radians).
\item{\texttt{b}}: Input.  The Galactic latitude~$b$ (radians).
\item{\texttt{alpha}}: Output.  The right ascension~$\alpha_{1950}$ (radians).
\item{\texttt{delta}}: Output.  The declination~$\delta_{1950}$ (radians).
\end{description}

The transformation is the following:
\begin{equation}
  \delta = \arcsin[
    \cos b\, \cos\delta_{\scriptstyle\mathrm{NGP}}\,
    \sin(\ell - \ell_{\scriptstyle\mathrm{ascend}}) + \sin b\,
    \sin\delta_{\scriptstyle\mathrm{NGP}} ]
\end{equation}
and
\begin{equation}
  \alpha = \arctan\left[
    \frac{ \cos b\, \cos(\ell - \ell_{\scriptstyle\mathrm{ascend}}) }
         { \sin b\, \cos\delta_{\scriptstyle\mathrm{NGP}} -
           \cos b\, \sin\delta_{\scriptstyle\mathrm{NGP}}\,
           \sin(\ell - \ell_{\scriptstyle\mathrm{ascend}}) } \right]
    + \alpha_{\scriptstyle\mathrm{NGP}}
\end{equation}
where
\begin{center}
\begin{tabular}{cl}
  $\alpha_{\scriptstyle\mathrm{NGP}}=192\degdot25$ &
    is the right ascension (1950) of the North Galactic Pole \\
  $\delta_{\scriptstyle\mathrm{NGP}}=27\degdot4$ &
    is the declination (1950) of the North Galactic Pole \\
  $\ell_{\scriptstyle\mathrm{ascend}}=33^\circ$ &
    is the ascending node of the Galactic plane on the equator. \\
\end{tabular}
\end{center}

\begin{description}
\item{Author:} Jolien Creighton, jolien@tapir.caltech.edu
\item{Comments:}  This routine is adapted from the method given in:
  Peter Duffet-Smith \emph{Practical Astronomy with Your Calculator,}
   3rd edition,
  (Cambridge University Press, 1988).  The values used in this routine should
  be updated to epoch 2000 coordinates.
\end{description}


\clearpage
\subsection{Example: \texttt{galactic2equatorial} program}
\label{ss:galactic2equatorial}

This is a simple implementation of the function
\texttt{galactic\_to\_equatorial()} that converts a specified Galactic
longitude and latitude to right ascension and declination (epoch 1950).
The Galactic coordinates are entered in decimal degrees as
command line arguments.  Some examples:
\begin{enumerate}
\item The North Galactic pole is $b=90^\circ$, or
$\alpha_{1950}=12^{\mathrm{\scriptstyle h}}49^{\mathrm{\scriptstyle m}}$
and $\delta_{1950}=+27^\circ24'$.  The output of the command
\verb|galactic2equatorial 0 90| is
\begin{verbatim}
l   (deg):   0.00
b   (deg): +90.00
RA  (hms):  12 48 59
Dec (dms): +27 23 59
\end{verbatim}
\item The Galactic center is $\ell=b=0$, or
$\alpha_{1950}=17^{\mathrm{\scriptstyle h}}42^{\mathrm{\scriptstyle m}}
  24^{\mathrm{\scriptstyle s}}$ and $\delta_{1950}=-28^\circ55'$.  The output
of the command \verb|galactic2equatorial 0 0| is
\begin{verbatim}
l   (deg):   0.00
b   (deg):  +0.00
RA  (hms):  17 42 26
Dec (dms): -28 55 00
\end{verbatim}
\item The Large Magellanic Cloud is $\ell=280\degdot5$
and $b=-32\degdot9$, or
$\alpha_{2000}=05^{\mathrm{\scriptstyle h}}23^{\mathrm{\scriptstyle m}}
  34\secdot598$ and
$\delta_{2000}=-69^\circ45'22\arcsecdot33$.
The output of the command \verb|galactic2equatorial 280.5 -32.9| is
\begin{verbatim}
l   (deg): 280.50
b   (deg): -32.90
RA  (hms):  05 23 48
Dec (dms): -69 49 36
\end{verbatim}
\end{enumerate}
These examples show that the output is correct to the accuracy of the
input coordinates.

\begin{description}
\item[author:] Jolien Creighton, jolien@tapir.caltech.edu
\end{description}

\clearpage
\lgrindfile{Includes/galactic2equatorial.tex}

\clearpage
\subsection{Function: \texttt{equatorial\_to\_horizon()}}
\label{ss:equatorial_to_horizon}

\begin{verbatim}
void equatorial_to_horizon(float alpha, float delta, float time, float lat,
			   float *azi, float *alt)
\end{verbatim}

This routine converts the coordinates of an object from the equatorial
system---right ascension~$\alpha$ and declination~$\delta$---to the horizon
system---azimuth~$A$ and altitude~$a$---for a given time and latitude.
The arguments are:
\begin{description}
\item{\texttt{alpha}}: Input.  The right ascension~$\alpha$ (radians).
\item{\texttt{delta}}: Input.  The declination~$\delta$ (radians).
\item{\texttt{time}}: Input.  The time of day (sidereal seconds).
\item{\texttt{lat}}: Input.  The latitude~$\lambda$ North (radians).
\item{\texttt{azi}}: Output.  The azimuth~$A$ (radians clockwise from North).
\item{\texttt{alt}}: Output.  The altitude~$a$ (radians up from the horizon).
\end{description}

The transformation is the following:
\begin{equation}
  a = \arcsin[ \sin\delta\, \sin\lambda + \cos\delta\, \cos\lambda\, \cos h ]
\end{equation}
and
\begin{equation}
  A = \arctan\left[ \frac{-\cos\delta\, \cos\lambda\, \sin h}
                         {\sin\delta - \sin\lambda\, \sin a} \right]
\end{equation}
where $\lambda$ is the latitude and $h=\mbox{(local sidereal time)}-\alpha$ is
the hour angle.

\begin{description}
\item{Author:} Jolien Creighton, jolien@tapir.caltech.edu
\item{Comments:}  This routine is adapted from the method given in:
  Peter Duffet-Smith \emph{Practical Astronomy with Your Calculator,}
  3rd edition,
  (Cambridge University Press, 1988).  I have assumed that the user either
  (a) has correctly precessed the equatorial coordinates, or (b) doesn't care.
\end{description}


\clearpage
\subsection{Function: \texttt{beam\_pattern()}}
\label{ss:beam_pattern}

\begin{verbatim}
void beam_pattern(float theta, float phi, float psi, float *plus, float *cross)
\end{verbatim}

This routine computes the beam pattern functions, $F_+$ and $F_\times$,
for some specified angles $\theta$, $\phi$, and $\psi$.  The arguments are:
\begin{description}
\item{\texttt{theta}}: Input.  The polar angle~$\theta$ (radians from zenith).
\item{\texttt{phi}}: Input.  The azimuthal angle~$\phi$ (radians
  counter-clockwise from the first arm).
\item{\texttt{psi}}: Input.  The polarization angle~$\psi$ (radians).
\item{\texttt{plus}}: Output.  The detector response function~$F_+$.
\item{\texttt{cross}}: Output.  The detector response function~$F_\times$.
\end{description}

The beam pattern functions are calculated according to the following formulae:
\begin{equation}
  F_+ = {\textstyle\frac{1}{2}}(1+\cos^2\theta)\cos2\phi\, \cos2\psi
    - \cos\theta\, \sin2\phi\, \sin2\psi
\end{equation}
and
\begin{equation}
  F_\times = {\textstyle\frac{1}{2}}(1+\cos^2\theta)\cos2\phi\, \sin2\psi
    + \cos\theta\, \sin2\phi\, \cos2\psi.
\end{equation}

\begin{description}
\item{Author:} Jolien Creighton, jolien@tapir.caltech.edu
\item{Comments:}  The beam pattern formulae, as well as a precise definition
  of the angles can be found in:
  Kip Thorne in \emph{300 Years of Gravitation}, S.\@ Hawking and W.\@ Israel
  editors (Cambridge University Press, 1987).  The formulae are suitable for
  detectors in which the arms are perpendicular; they are not suitable for the
  GEO-600 site because the opening angle is approximately $94^\circ$.
\end{description}


\clearpage
\subsection{Function: \texttt{mc\_chirp()}}
\label{ss:mc_chirp}

\begin{verbatim}
void mc_chirp(float time, float latitude, float orientation, long *seed,
	      float *invMpc, float *c0, float *c1)
\end{verbatim}

This routine makes a random chirp from a hypothetical distribution of Galactic
binary neutron stars.  The arguments are:
\begin{description}
\item{\texttt{time}}: Input.  The time of arrival of the chirp in sidereal
  seconds.
\item{\texttt{latitude}}: Input.  The latitude of the detector in radians
  North.
\item{\texttt{orientation}}: Input.  The orientation of the arm of the detector
  in radians counter clockwise from North.
\item{\texttt{seed}}: Input/Output.  A random number seed for use in the
  Numerical Recipes random number generator \texttt{ran1()}.
\item{\texttt{invMpc}}: Output.  The effective distance of the chirp, i.e.,
  the distance of the binary system if it were optimally oriented.
\item{\texttt{c0}}: Output.  The cosine of the random phase of the chirp.
\item{\texttt{c1}}: Output.  The sine of the random phase of the chirp.
\end{description}
The resulting chirp waveform is
$h(t)=\texttt{invMpc}\times[\texttt{c0}\times h_c(t)+\texttt{c1}\times h_s(t)]$
where $h_c(t)$ is the cosine-phase chirp waveform and $h_s(t)$ is the
sine-phase chirp waveform normalized to one megaparsec.

The hypothetical number of Galactic binary neutron star systems between the
Galactocentric radii $R$ and $R+dR$, and between disk heights $z$ and $z+dz$,
is taken to be
\begin{equation}
  dN \propto R\,dR\,dz\,e^{-R^2/2R_0^2}\,e^{-|z|/h_z}
\end{equation}
with $R_0=4.8\:\mbox{kpc}$ and $h_z=1\:\mbox{kpc}$.  Only the disk population
is considered in this calculation.  The Galactocentric radius of the Sun
is taken to be $R_\odot=8.5\:\mbox{kpc}$.

\begin{description}
\item{Author:} Jolien Creighton, jolien@tapir.caltech.edu
\item{Comments:} The hypothetical distribution was adapted from the
  distribution used by S.\@ J.\@ Curran and D.\@ R.\@ Lorimer,
  Mon.\@ Not.\@ R.\@ Astron.\@ Soc.\@ \textbf{276} 347 (1995).  The scale
  height is a guestimate.  Very little is known about the actual distribution
  of Galactic neutron star binaries!  The routine assumes that the arms of the
  detector are perpendicular; it is not suitable for the
  GEO-600 site because the opening angle is approximately $94^\circ$.
\end{description}
\clearpage


\subsection{Example: \texttt{inject} program}
\label{ss:insert}
This program produces a list of random Galactic neutron star-neutron star
binary inspiral
parameters for injection into the interferometer data to test the data
analysis software.

The inspiral waveforms correspond to two neutron star companions, each with a
mass distribution that is uniform between two cutoffs given by the parameters
\texttt{MLO} and \texttt{MHI}
[for example see L S Finn, Phys Rev Lett \textbf{73} 1878 (1994)].

The amplitude distribution corresponds to the (time dependent) model
described for the GRASP routine \texttt{mc\_chirp()}.  The initial phase is
uniformly distributed.  The parameters \texttt{LAT}, \texttt{LON}, and
\texttt{ARM} correspond to the latitude, longitude, and arm orientation of the
detector, and are required for the model.

The injection time is either at a fixed intervals given by parameter
\texttt{INV\_RATE}
(when parameter \texttt{FIXED}=1), or at random intervals corresponding to a
Poisson process with inverse rate \texttt{INV\_RATE} (when \texttt{FIXED}=0).
Injection times are between the start and the end of the data run specified
by the environment variable \texttt{GRASP\_DATAPATH}.  The start and end times
of this data run are obtained from code resembling that in program
\texttt{locklist}.  If two chirps potentially occur within the same data
segment (with length given by parameter \texttt{NPOINT}), a warning message is
printed.

The results are output to stdout in a list containing the arrival time
(double), the two masses (floats), the amplitude---inverse Mpc distance
(float), and the initial phase (float), separated by spaces.  This is the
same format as required for the file \texttt{insert.ascii} which is read by
the \texttt{binary\_get\_data()} routine in the \texttt{binary\_search} code.

\begin{description}
\item[Author:] Jolien Creighton, jolien@tapir.caltech.edu
\end{description}

\clearpage
\lgrindfile{Includes/inject.tex}


