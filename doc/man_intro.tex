% GRASP: Copyright 1997,1998,1999  Bruce Allen
% $Id: man_intro.tex,v 1.52 2000/05/19 15:19:52 ballen Exp $
\section{Introduction}
\setcounter{equation}0
\subsection{The Purpose of GRASP}
\setcounter{equation}0
The analysis and modeling of data from gravitational wave detectors
requires specialized numerical techniques.  GRASP was developed in
collaboration with the Laser Interferometer Gravitational Wave Observatory
(LIGO) project in the United States, and contains a collection of
software tools for this purpose.
The first release of GRASP was in early 1997; since that time many individuals have made extensive
contributions.

In order that it be of the most use to the physics community,
this package (including all source code) is being released in the
public domain.  It may be freely used for any purpose, although we do
ask that GRASP and its author be acknowledged or referenced in
any work or publications to which GRASP made a contribution.
If possible this reference should include a link to the GRASP distribution
web site:\\
\htmladdnormallink{{\tt http://www.lsc-group.phys.uwm.edu/$\sim$ballen/grasp-distribution/}}
{http://www.lsc-group.phys.uwm.edu/~ballen/grasp-distribution}.
The
citation should specify the {\it version number} (for example, 1.9.1) of
GRASP.  In addition, if the code has been modified please state this.
We suggest that if GRASP is installed at a site,
one person at the institution should be designated as the
``responsible party" in charge of the GRASP package.

GRASP is intended for a broad audience, including those users whose
main interest is in running simulations and analyzing data, and those
users whose main interest is in testing new data analysis techniques or
incorporating searches for new types of gravitational wave sources.
The GRASP package includes a ``cookbook" of documented and tested
low-level routines which may be incorporated in user code, and simple
example programs illustrating the use of these routines.  GRASP also
includes a number of high level user applications built from these
routines.

We are always interested in extending the capabilities of GRASP.
Suggestions for changes or additions, including reports of bugs or
corrections, improvements, or extensions to the source code, should be
communicated directly to the author.

\subsection{Printing/Reading the Manual}
\setcounter{equation}0
The manual is distributed with GRASP in three forms.  In the GRASP
directory {\tt doc} you can find a Portable Document Format (PDF) file
{\tt manual.pdf}, a Postscript file {\tt manual.ps} and a Device Independent
file {\tt manual.dvi}.  We suggest using the PDF file.  Not only is it
compact, but all the sections, references, and equations are represented
as clickable links.  Even the WWW links (URLs) can be clicked on and will fire
up your favorite Web browser.  You can also easily ``zoom-in" on interesting
graphs.

If you want a printed copy of the manual, there are two options.
We find that the most readable form is ``2-up".  You can make a
postscript file of this form using the {\tt psnup} utility, available as
part of the public-domain package {\tt psutils}.  Use the commands:\\
{\tt
psnup -2 /usr/local/GRASP/doc/manual.ps man2.ps\\
}
and then print the file that you have just created ({\tt man2.ps}) on a two-sided postscript printer.  You'll end up with four pages of this
manual on a single sheet of paper.

If you want a copy of the manual with the color graphs in color rather than gray-scale,
we've included postscript files containing
the color and black-and-white pages separately.  Print
{\tt doc/manual\_color.ps} on a color printer, and {\tt doc/manual\_bw.ps}
on a black and white printer, and start collating!

\subsection{Quick Start}
\setcounter{equation}0
If you hate to read manuals, and you just want to try something, here's
a suggestion.  This assumes that the GRASP package has been installed
by your local system administrator in a directory accessible to you,
such as {\tt /usr/local/GRASP} and that some 40-meter data (old-format)
has also been installed, for example in {\tt /usr/local/GRASP/data}.

If you want to try running a GRASP program, type\\
{\tt setenv GRASP\_DATAPATH /usr/local/GRASP/data/19nov94.3}\\
to set up a path to the data, then go to the GRASP directory:\\
{\tt cd /usr/local/GRASP/src/examples/examples\_40meter}\\
and try running one of the executables:\\
{\tt ./locklist}\\
will print out a list of the locked data segments from run 3 on 19 November 1994.
A more interesting program to run (in the same directory) is\\
{\tt ./animate | xmgr -pipe}\\
which will produce an animated display of the IFO output.  Note that in
order for this to work, you will need to have the {\tt xmgr} graphing
program in your path.  (Please see the comment about {\tt xmgr} in
Section~\ref{s:animate}).

If you only have data that has been distributed in the FRAME format,
type\\
{\tt setenv GRASP\_FRAMEPATH /usr/local/GRASP/data/19nov94.3.frame}\\
to set up a path to the data, then go to the GRASP directory:\\
{\tt cd /usr/local/GRASP/src/examples/examples\_frame}\\
and try running one of the executables:\\
{\tt ./locklistF}\\
will print out a list of the locked data segments from run 3 on 19 November 1994.
A more interesting program to run (in the same directory) is\\
{\tt ./animateF | xmgr -pipe}\\
which will produce an animated display of the IFO output.  Note that in
order for this to work, you will need to have the {\tt xmgr} graphing
program in your path.  (Please see the comment about {\tt xmgr} in
Section~\ref{s:animate}).

If you want to try writing some GRASP code, a simple way to start is to 
copy one of the example programs, and the Makefile, into your personal directory,
and edit that:\\
{\tt mkdir $\sim$/GRASP}\\
{\tt cp /usr/local/GRASP/src/examples/examples\_40meter/gwoutput.c $\sim$/GRASP}\\
{\tt cp /usr/local/GRASP/src/examples/examples\_40meter/Makefile $\sim$/GRASP}\\
{\tt cd $\sim$/GRASP}\\
Now make editing changes to the file {\tt gwoutput.c}, and when you are
done, edit the {\tt Makefile} that you have
copied into your home directory.  Find the line that reads:\\
{\tt all: ... gwoutput ...}\\
and delete everything to the right of the colon except {\tt gwoutput} from that
line (but leave a space after the colon).  Then type:\\
{\tt make gwoutput}\\
to recompile this program.  To run it, simply type:\\
{\tt gwoutput}.\\
In general, if you want to modify GRASP programs, this is the simplest way to start.

\subsection{A few words about data formats}
\label{ss:dataformats}
\setcounter{equation}0
The GRASP package was originally written for analysis of data in the
``old" format, which was used in the Caltech 40-meter IFO laboratory
prior to 1996.  Starting in 1997, the LIGO project, and a number of other
gravity-wave detector groups, have adopted the VIRGO FRAME data format.
Almost every example in the GRASP package has equivalent programs to
read and analyze data in either format.  For example {\tt animate}
and {\tt animateF} are two versions of the same program.  The first
reads data in the old format, the second reads data in the FRAME format.
We have also included with GRASP a translation program that translates
data from the old format to the new format (see {\tt translate} in
Section~\ref{ss:translate}).

After careful thought, the LIGO management has decided to only distribute
the November 1994 data in the FRAME format, except to a small number of
groups (belonging to the {\it Data Translation Group}) who are responsible
for ensuring that the translated data set contains the same information
as the original!  The initial distributions of GRASP will include both
old-format and new-format code.  However after a reasonable period of
time, the old-format data and code will be removed from the package.
So please be aware that the old-format material will be reaching the
end of its useful lifetime fairly soon; we do not recommend investing
much effort in these.

If you want to develop or work on data analysis algorithms, you will
want to have access to this data archive.  Because many people
contributed to taking this data, and because the LIGO project wants to
maintain control of its use and distribution, {\it this data set is NOT
in the public domain}.  However, you may request a copy for your use,
or for use by your research group.  Write to: Director of the LIGO
Laboratory, Mail Stop 51-33, California Institute of Technology,
Pasadena, CA 91125.  The data set is available in {\tt tar} format on
two Exabyte 8500c format tapes.

In order to use the data in the FRAME format, you will need to have access
to the FRAME libraries.  These are available from the VIRGO project; they
may be downloaded from the site\\
\htmladdnormallink{{\tt http://wwwlapp.in2p3.fr/virgo/FrameL/}.}
{http://wwwlapp.in2p3.fr/virgo/FrameL/}
The current release of GRASP is compatible with versions of the FRAME library
$\le 3.72$.
Contact Benoit Mours {\tt mours@lapp.in2p3.fr} for further information.

\subsection{GRASP Hardware \& Software Requirements}
\setcounter{equation}0
GRASP was developed under the Unix (tm) operating system, on a Sun
workstation network.  The package is written in POSIX/ANSI C, so that
GRASP can be compiled and used on any machine with an ANSI C compiler.
All operating system calls are POSIX-compliant, which is intended to keep
GRASP as portable to different platforms as possible.  The main routines
could also be linked to user code written in other languages such as
Fortran or Pascal; the details of this linking, and the conventions
by which Fortran and C (or Pascal and C) routines communicate are
implementation dependent, and not discussed here.

Several of the high-level applications in GRASP can be run on parallel
computer systems.  These can be either dedicated parallel computers
(such as the Intel Paragon or IBM SP2 machines) or a network of
scientific workstations.  The parallel programming in GRASP is
implemented with version 1.1 of the Message Passing Interface (MPI)
library specification \cite{MPI}.  All major computer system vendors
currently support this standard, so GRASP can be easily compiled and
used on virtually any parallel machine.  In addition, there is a
public-domain implementation of MPI called ``mpich" \cite{MPICH} which
will run MPI-based programs on networks of scientific workstations.
This makes it easy to do ``super-computing at night" by running GRASP on
a network of workstations.  Further information on MPI is available
from the web site\\
\htmladdnormallink{{\tt http://www.mcs.anl.gov/mpi/}.}
{http://www.mcs.anl.gov/mpi/}
The mpich
implementation is available from\\
\htmladdnormallink{{\tt http://www.mcs.anl.gov/mpi/mpich/}.}
{http://www.mcs.anl.gov/mpi/mpich/}
By the way, if you don't have access to parallel machines (or have no
interest in parallel computing) don't worry!  The only parallel code in
GRASP is found in ``top-level" applications; all of the functions in
the GRASP library, and most of the examples, can be used without any
modifications on a single processor, stand-alone computer.

GRASP makes use of a number of standard numerical techniques.  In general,
we use version 2.06 of the routines from ``{\it Numerical Recipes in
C: the art of scientific computing}" \cite{NumRec}.  [Later versions
should work OK -- please let me know if they don't.]  These routines
are widely used in the scientific community.  The full source code,
examples, and complete documentation are provided in the book,
and are also available (for about \$50) in computer readable form.
Ordering information and further details are available from\\
\htmladdnormallink{{\tt http://www.nr.com/}.}
{http://www.nr.com/}
These routines are
extremely useful and beautifully-documented; if you don't already have
them available for your use, you should!

Certain routines in that use inter-channel correlations to `clean' a
signal channel also use CLAPACK numerical linear algebra libraries. 
These are extremely robust and well tested libraries and are an
extremely valuable complement to {\it Numerical Recipes}.  
Note that all GRASP programs can be compiled without CLAPACK but that
some inter-channel correlation functions will not be available without
it. The full source code 
for these may be downloaded from
\htmladdnormallink{{\tt http://www.netlib.org/clapack/}}{http://www.netlib.org/clapack/}.

The time-frequency routines in the GRASP package also come with a function 
({\tt plottf()}) to display time 
frequency-maps on the screen using calls to the MESA graphics library. 
This library is a GL lookalike and available freely
from \htmladdnormallink{{\tt http://www.mesa3d.org/}}{http://www.mesa3d.org/}.

In general, output from GRASP is in the form of ASCII text files.
We assume that the user has graphing packages available to
visualize and interpret this output.   Our personal favorite is
{\tt xmgr}, available in the public domain from the site
\htmladdnormallink{{\tt http://plasma-gate.weizmann.ac.il/Xmgr/}}
{http://plasma-gate.weizmann.ac.il/Xmgr/}
which also lists mirror
sites in Europe and USA. (Please see the comment about {\tt xmgr} in
Section~\ref{s:animate}).  In some cases we do output ``complete graphs"
for {\tt xmgr}.  We do also output some data in the form of PostScript
(tm) files.  Previewers for postscript files are widely available in
the public domain (we like GhostView).

\subsection{GRASP Installation}
\setcounter{equation}0
As we have just explained, GRASP requires access to {\it Numerical
Recipes in C} libraries and to MPI and MPE libraries and optionally
to the CLAPACK libraries. 
These packages must be installed, and then within GRASP a path to
these libraries must be defined.  This can be done by editing a single
file, and then running a shell script.  This section explains each of
these steps in detail.

All of the site-specific information is contained in a single file {\tt
SiteSpecific} in the top-level directory of GRASP.  This file contains
a number of variables whose purpose is explained in this section.
These variables must be correctly set before GRASP can be used; the
definitions contained in  {\tt SiteSpecific} (as distributed) are
probably {\it not} appropriate for your system, and will therefore
require modification.  A number of example {\tt SiteSpecific} files
are included in the GRASP distribution, in the directory {\tt Examples\_SiteSpecific/ }.

\subsubsection{GRASP File Structure}
The code for GRASP can be installed in a publicly-available directory,
for example \mbox{\tt /usr/local/GRASP}.  (It can also be installed
``privately" in a single user's home directory, if desired.)  The name of
this top-level directory must be set in the file {\tt SiteSpecific} which
is contained in the top-level GRASP directory.  To do this, edit the file
{\tt SiteSpecific} and set the variable \mbox{\tt GRASP\_HOME}  to the
appropriate value, for example \mbox{\tt GRASP\_HOME=/usr/local/GRASP}.
Please note that the installation scripts are not designed to ``build"
in one location and ``install" in a separate location.  You should go through
the installation procedure in the same directory where you eventually want
the GRASP package to reside.

Within this top level directory resides the entire GRASP package.  The
directories within this top level are:
\begin{description}
\item[{\tt Examples\_SiteSpecific}] Contains examples of SiteSpecific
files for different sites, machine-types, and installations.  You may
find this helpful in the installation process if you want to look at
an example, or you are stuck.
\item[{\tt bin/}] Contains links to all the example programs and
scripts in the GRASP package.
\item[{\tt data/}] Contains (both real and simulated) interferometer data,
  or symbolic links to this data.  See the comments in Section~\ref{s:40meter}
  to find out how to obtain this data.
\item[{\tt doc/}] Documentation (in TeX, PostScript, DVI, and PDF formats)
   including this users guide.
\item[{\tt include/}] Header files used to define structures and other
   common types in the code.
This also include the ANSI C prototypes for all the GRASP functions.
\item[{\tt lib/}] Contains the GRASP library archive: {\tt
libgrasp.a}. To use any of the GRASP functions within your own code,
simply link this library with you own code.
\item[{\tt man/}] This may be used in the future for UNIX on-line manual pages.
\item[{\tt parameters/}] Contains parameters such as site location information,
  and estimated power spectra and whitening functions of future detectors.
\item[{\tt src/}] Source code for analyzing various aspects of the
data stream, distributed among the following directories:
\begin{description}
\item{\tt 40-meter/} Reading data tapes produced on the Caltech 40 meter
  prototype prior to 1997.
\item{\tt GRtoolbox/} Source code for the Gravitational Radiation Toolbox, a
Matlab (command line and GUI) interface to GRASP.
\begin{description}
\item{\tt mexfiles/}
Mex-files for use with the Gravitational Radiation Toolbox.
\end{description}
\item{\tt examples/} The source code for all of the examples given in
this manual (organized by section).  These include:
\begin{description}
\item{\tt examples\_40meter/} Examples of reading/using old-format
40-meter data.
\item{\tt examples\_GRtoolbox/} M-file examples for the Gravitational 
Radiation Toolbox.
\item{\tt examples\_binary-search/} The source code and documentation for a
binary-inspiral search carried out on the Caltech 40-meter data from
November 1994.
\item{\tt examples\_correlation/} Examples of determining correlations
between different channels and using the knowledge of these
correlations to `clean up' a particular channel.
\item{\tt examples\_frame/} Examples of reading/using new-format FRAME
data.
\item{\tt examples\_galaxy/} Examples of using galactic models to
predict source distribution parameters.
\item{\tt examples\_inspiral/} Examples of generating inspiral
waveforms and searching for them in the data stream using matched filtering.
\item{\tt examples\_ringdown/} Examples of generating
black-hole-horizon formation ringdown waveforms and searching for them
in the data stream using matched filtering.
\item{\tt examples\_stochastic/} Examples of simulated production of a
stochastic background correlated signal between two detector sites and
a pipeline to search the data stream for such signals.
\item{\tt examples\_template\_bank/} Example code for setting up a
bank of binary-inspiral templates and graphing their locations in
parameter space.
\item{\tt examples\_testmass/} Example code for evaluating binary inspiral
waveforms in the testmass limit $m_1 \to 0$ and comparing the resulting
waveforms with those calculated by other methods.
\item{\tt examples\_timefreq/} Example code illustrating the use of time-frequency techinques
for signal detection.
\item{\tt examples\_transient/} Example code to generate and search
for transient waveforms such as those arising from supernovae.
\item{\tt examples\_utility/} Examples of various utility functions,
including a translator to produce new-format FRAME data from old
format 40-meter data.
\end{description}
\item{\tt correlation/} Code for calculating correlations between
 different channels and `cleaning' a particular channel. 
\item{\tt galaxy/} Modelling the distribution of sources in our galaxy
(needed in order to set physical upper-limits using the 40-meter
prototype data).
\item{\tt inspiral/}  Binary inspiral analysis (including optimal filtering and
vetoing).
\item{\tt optimization/} Additional library routines for optimizing GRASP operation of specific platforms (i.e., supercomputers).
\item{\tt ringdown/}  Black hole horizon ringdown (including optimal filtering).
  This can be used to filter for {\it any} exponentially-decaying sinusoid.
\item{\tt stochastic/} Stochastic background detection (including optimal filtering and simulated signal production)
\item{\tt transient/}  Supernovae and other transient sources.
\item{\tt periodic/} Searches for pulsars and other periodic and
quasi-periodic sources.
\item{\tt template\_bank/} Code for ``placing'' optimal filters in
parameter space.
\item{\tt testmass/} Code for calculating binary inspiral waveforms in the test mass
limit $m_1 \to 0$.
\item{\tt timefreq/} Code for time-frequency transforms, and searching for line-like
features in the time-frequency maps.
\item{\tt utility/} General purpose utility routines, including the
interface to the FRAME library, error handler routines, etc.
\end{description}
\item[{\tt testing/}] This will eventually contain a suite of programs
that test the GRASP installation.
\end{description}


\subsubsection{Accessing {\it Numerical Recipes in C} libraries}
GRASP makes use of many of the functions and subroutines from {\it
Numerical Recipes in C} \cite{NumRec}.
The web site \htmladdnormallink{{\tt http://www.nr.com/}}
{http://www.nr.com/} is a good source of further information.
These functions and subroutines
are available in Fortran, Pascal, Basic, Kernighan and Ritchie (K\&R)
C, and ANSI-C versions; you will need the ANSI-C routines.  The source
code for these functions (both {\tt *.c} and {\tt *.h} files) must be
installed in a directory (for example, \mbox{\tt /usr/local/recipes/src})
and the compiled object modules ({\tt *.o} files) must be archived into
a single library file ({\tt *.a} file).  The instructions for this are
included in the distribution of the source code for {\it Numerical
Recipes}.   In the end, a file called {\tt librecipes\_c.a} must be put
into a directory where it is available to the linker for compilation.
A good place to put this library is in \mbox{\tt
/usr/local/recipes/lib/librecipes\_c.a}.  When you run the command that
installs GRASP, the linker needs to be able to find these libraries.
The file {\tt SiteSpecific} must then contain the line \mbox{\tt
RECIPES\_LIB = /usr/local/recipes/lib} near the top of the file.

It is frequently useful, for debugging purposes, to be able to link
with both ``debug" and ``profile" versions of the libraries.  For this
reason, we recommend that users actually create {\it three separate
libraries} of {\it Numerical Recipes} functions:
\begin{description}
\item{\tt /usr/local/recipes/lib/librecipes\_c.a:}
a library compiled for fast execution, with optimization options (for
example, -O3 or -xO4) turned on during compilation.
\item{\tt /usr/local/recipes/lib/librecipes\_cg.a:}
a library compiled for debugging, with the debug option (typically, -g)
turned on during compilation.  Note that in order to use a debugger
with this library, and to be able to step ``within" the {\it Numerical
Recipes} functions, the debugger must be able to locate the source code
for {\it Numerical Recipes}.  Thus, after {\it Numerical Recipes} is
compiled and installed, its *.c and *.h source files must be left in
their original locations and not deleted or moved.
\item{\tt /usr/local/recipes/lib/librecipes\_cp.a:}
a library compiled for profiling, with the profiling option (typically,
-pg or -xpg for ``gprof" or -p for ``prof") turned on during compilation.
\end{description}
One can then easily compile GRASP code with the appropriate library by
setting {\tt LRECIPES} in {\tt SiteSpecific}.  For example to run code
as rapidly as possible one would set \mbox{\tt LRECIPES = recipes\_c}.
However to compile code for debugging it would be preferable to set
\mbox{\tt LRECIPES = recipes\_cg}.  (Note that rather than recompiling the
entire GRASP package in this way, one can simplify modify the value of
{\tt LRECIPES} within the desired {\tt Makefile}s and then recompile
only the code of interest.)

We have encountered one minor problem with the {\it Numerical Recipes
in C} routines.  Unfortunately the authors of these routines choose to
name one of their routines {\tt select()}.  This name conflicts with a
POSIX name for one of the standard operating system calls.  In linking
with certain libraries (for example the MPI/MPE libraries) this can
generate conflicts where the linker attaches the {\tt select()} call to
the entry point from the wrong library.  
Starting with release 1.6.3 of GRASP, the {\tt select()} routine from
Numerical Recipes is used in GRASP.  For this reason, you must
fix this
as follows.  Before building the {\it Numerical Recipes} libraries, edit
the source files {\tt recipes/rofunc.c}, {\tt recipes/select.c}, and {\tt
recipes/select.c.orig} changing each occurence of {\tt select(} to {\tt
NRselect(}.  You will have to do this in (respectively) four places, one
place and one place in these files.  Then edit the file {\tt include/nr.h}
making the same change of {\tt select(} to {\tt NRselect(} in one place.
This will elminate the {\tt select()} routine from the {\it Numerical
Recipes} library, replacing it with a routine called {\tt NRselect()},
and eliminating any possible naming conflict from the library.
So, to summarize, the routine called {\tt select()} in the {\it Numerical
Recipes} library is used in GRASP, but is called {\tt NRselect()} there.

\subsubsection{Accessing MPI and MPE libraries}
To enable use of the parallel processing code included with GRASP, one
needs to link the code with an MPI function call library.  (If you do not
intend to use any of the multiprocessing code, we'll tell you what to do.)
For performance monitoring purposes, we also make calls to the Message
Passing Environment (MPE) library, which is included with {\tt mpich}
\cite{MPICH}.  If these function libraries are not currently available
on your system, you should obtain the public domain implementation {\tt
mpich} from the URL\\
\htmladdnormallink{{\tt http://www.mcs.anl.gov/mpi/mpich/}}
{http://www.mcs.anl.gov/mpi/mpich/}
and follow the instructions required
to build the MPI/MPE libraries for your system.  After the installation
process is complete, the necessary libraries will be contained in a
library archive, for example \mbox{\tt /usr/local/mpi/lib/libmpi.a} and
\mbox{\tt /usr/local/mpe/lib/libmpe.a}.  The path to these libraries
is set in the file {\tt SiteSpecific} by means of the variable {\tt
MPI\_LIBS}.  A typical line in {\tt SiteSpecific} might then read:\\
\mbox{\tt MPI\_LIBS=-L/usr/local/mpi/lib -lmpi -lmpe}.\\
You must also set {\tt BUILD\_MPI= true } in {\tt SiteSpecific}.
Finally, in order to include appropriate header files in any MPI programs,
you will need to include a path to these header files in the file {\tt
SiteSpecific}.  You can do this by setting {\tt MPI\_INCLUDES} in the
file {\tt SiteSpecific}.  A typical installation might have \\
{\tt MPI\_INCLUDES = -I/usr/local/mpi/include}.\\
NOTE: If you don't want to use {\it any} of the MPI code, just set:\\
{\tt BUILD\_MPI= false}\\ in {\tt SiteSpecific}.  All the other
MPI-specific defines are then ignored.  
 
\subsubsection{Accessing {\it MESA} libraries}
Currently one of the routines available in GRASP, {\tt plottf()},
requires the Mesa library to display the time-frequency maps on the
screen. 
Mesa is a 3-D graphics library with an 
API which is very similar to
that of OpenGL. Mesa is distributed under the terms of
the GNU Library General Public License. The Mesa library 
may be downloaded from
\htmladdnormallink{{\tt http://www.mesa3d.org/}}.
If you are not interested in using the {\tt plottf()}
routine, you may set {\tt HAVE\_GL= false} in {\tt SiteSpecific} and
ignore the rest of this section.


The Installation is extremely simple. Download the file 
{\tt MesaLib-3.0.tar.gz}. The ungzipped untarred 
file produces a directory tree under {\tt Mesa-3.0}. Enter the 
directory {\tt Mesa-3.0}, and key in {\tt make}. This lists a variety
of systems on which the Mesa library has been compiled. Select the one
which most accurately describes your system and key in, 
{\tt make my\_system},
where {\tt my\_system} is what you have selected from the list. 
This will compile the programs and create the Mesa libraries in the
directory, {\tt Mesa-3.0/lib}. Copy the libraries to a common location
such as {\tt /usr/local/lib} and copy the include files in
{\tt Mesa-3.0/include-} to  a common location such as {\tt
/usr/local/include}. (The files {\tt README} and {\tt README.*} files
have detailed instructions to install the software, if required.)

\subsubsection{Accessing CLAPACK libraries}
As mentioned above GRASP uses routines form CLAPACK to perform 
the numerical linear algebra required in some of the environmental
correlation routines. The routines that require these libraries are
those which `clean up' one channel based on an analysis of the
correlations between a number of channels.
 In the case of the data stream from an
interferometric gravitational radiation detector, the primary interest 
would be in the cleaning the signal determining the differential
displacement  of suspended test masses using information from
environmental channels.  If you are not interested in such routines
you may set\\ {\tt WITH\_CLAPACK= false}\\ in {\tt SiteSpecific} and
ignore the rest of this section.

The CLAPACK routines may be downloaded from\\
\htmladdnormallink{{\tt http://www.netlib.org/clapack/}}
{http://www.netlib.org/clapack/}.  It is simplest to download the 
complete package
\linebreak[4]
{\tt clapack/clapack.tgz} although it is possible
to download individual elements if disk space is at a premium 
(the complete package includes testing and timing routines which may
be discarded after successful installation). The ungzipped untarred 
file produces a directory tree under CLAPACK.   The directory
 CLAPACK contains LAPACK make include file {\tt make.inc} where
compiler flags etc are set.  You may wish to change the lines
\begin{verbatim}
BLASLIB      = ../../blas$(PLAT).a
LAPACKLIB    = lapack$(PLAT).a
\end{verbatim}
to 
\begin{verbatim}
BLASLIB      = ../../libblas$(PLAT).a
LAPACKLIB    = liblapack$(PLAT).a
\end{verbatim}
CLAPACK uses the f2c libraries so the first step is to create these by
typing {\tt cd F2CLIBS/libF77; make} and {\tt cd F2CLIBS/libI77; make}
each time starting from the CLAPACK directory.  Next one builds the
BLAS (Basic Linear Algebra Subprograms) libraries with {\tt cd
BLAS/SRC; make}.  Finally one builds the CLAPACK library with {\tt cd
SRC; make}.  The f2c libraries {\tt libF77.a} and {\tt libI77.a} and 
include file {\tt f2c.h} are now in the subdirectory F2CLIBS of
CLAPACK while the libraries {\tt libblas\$(PLAT).a} and {\tt liblapack\$(PLAT).a}
are in the directory CLAPACK. From here they may be installed into appropriate
directories.  Fuller details, including the building and running of the
test and timing programs may be found in the {\tt README} file in the
CLAPACK directory.  As for the Numerical Recipes
libraries it can be convenient to have both optimised and debugging
versions of the libraries available for development work.

\subsubsection{Accessing FRAME libraries}
The LIGO and VIRGO detector projects have recently decided to
standardize the format which their data will be recorded in (see
Section~\ref{ss:dataformats}).  The standard is called the FRAME format,
and is still under development.  It appears quite possible that a number
of other gravitational-wave detector groups will also adopt this same
format.  The GRASP package contains, for every example program, both FRAME
format and old format versions.  It also contains an translation program
which converts data from the ``old 1994" format into the new FRAME format.

Unless you are in one of the small number of groups with access to
the old-format data, you will need to obtain the FRAME libraries.
These are available from the VIRGO project; they may be downloaded from
the site
\htmladdnormallink{{\tt http://wwwlapp.in2p3.fr/virgo/FrameL/}.}
{http://wwwlapp.in2p3.fr/virgo/FrameL/}
Contact Benoit
Mours
\linebreak[4]
{\tt mours@lapp.in2p3.fr} for further information.  In the {\tt
SiteSpecific} file, if you need the FRAME libraries, set a pointer to the
directory containing them.  NOTE: If you don't need the FRAME libraries,
just set:\\ {\tt BUILD\_FRAME = false}\\ in {\tt SiteSpecific}.  All the
other FRAME-specific defines are then ignored.

The GRASP interface to the FRAME library should work properly with
every version of the FRAME library from 2.30 onwards.  The GRASP
interface to the FRAME library looks to see which version of the FRAME
library you are using, and then generates the appropriate code.  The
FRAME library is designed to be backwards-compatible.  For example,
version 3.42 of the FRAME library can read files written with version
2.37 of the FRAME library. GRASP has been tested with versions of the FRAME
$\le 3.72$.


\subsubsection{Real-time 40-meter analysis}
The analysis tools in the GRASP package can be used to analyze data
in real-time, as it is recorded by the DAQ system.  This facility is
primarily for the use of experimenters working in the Caltech 40-meter lab.
and will probably not be of use to anyone outside of that group.

In order to use the GRASP tools in real time, one needs to link to
a set of EPICS (Experimental Physics and Industrial Control System)
libraries, that are not otherwise needed.  These permit the GRASP code
to interrogate the EPICS system to find out the names and locations of
the most-recently written FRAMES of data.

\subsubsection{The Matlab Interface}
\label{sss:GRtoolboxInstall}
The Gravitational Radiation Toolbox provides a
Matlab interface to both GRASP and the Frame Library.
The Gravitational Radiation Toolbox links these two packages with
Matlab---simultaneously exposing data to a familiar, commercially developed,
problem solving environment and efficient algorithms designed specifically for
analyzing gravitational radiation data.

\subsubsection{Making the GRASP binaries and libraries}
\label{ss:buildit}
To make the GRASP libraries and executables described in this manual,
please follow these directions.  It should only take a few minutes to
do this.
\begin{enumerate}
\item
Within the main GRASP directory is a file called {\tt SiteSpecific}.
Make a copy of {\tt SiteSpecific} called {\tt SiteSpecific.save}.  This
way, if you mess up the installation, you can start over easily.
(Alternatively, copy {\tt SiteSpecific} to a file called {\tt
SiteSpecific.mysite} and, everywhere below, when we refer to editing
the {\tt SiteSpecific} file, edit {\tt
SiteSpecific.mysite} instead.)  Note: you can find a number of example
{\tt SiteSpecific} files in the directory {\tt
Examples\_SiteSpecific/ }.
These are for different installation sites and machine types (Sun,
DEC, Intel Paragon, IBM SP2, Linux) -- you
may find them helpful if you are stuck or the instructions below are
ambiguous or unclear.  Once you have customized the {\tt SiteSpecific}
file for your own installation, if you wish you can email it to
us and we will include it in future releases of GRASP.
\item
Now edit {\tt SiteSpecific} so that {\tt GRASP\_HOME} has the correct
path, for example \\ 
\mbox{\tt GRASP\_HOME=/usr/local/GRASP}.\\  
This must be the
name of the directory on your system in which GRASP resides.  If you
are not the superuser and are installing GRASP only for your own use,
you can set this path to point somewhere in your own home directory,
and install GRASP there.
\item
Find out where {\it Numerical Recipes in C} is installed on your
system.  Within {\tt SiteSpecific} set {\tt RECIPES\_LIB} to point to
the directory containing these libraries.  For example\\
\mbox{\tt RECIPES\_LIB=/usr/local/numerical\_recipes/lib}.\\  If {\it Numerical
Recipes in C} is not installed on your system, you will have to obtain
a copy, and install it, following the directions to create the library
file {\tt librecipes\_c.a}.  Note that as described above, you might
also want to create debugging libraries {\tt librecipes\_cg.a} and
profiling libraries {\tt librecipes\_cp.a}.
\item
Within {\tt SiteSpecific} set {\tt LRECIPES} to the name of the {\it
Numerical Recipes in C} library you wish to use, for example \\
\mbox{\tt LRECIPES=recipes\_c}.
\item
If you intend to use the MPI code, set {\tt BUILD\_MPI= true}, otherwise
set it to {\tt false}.  In this latter case, any MPI-specific defines
are ignored, and no code that makes use of MPI/MPE function calls is
compiled.  (This is a shame -- these are some of the nicest programs in
the GRASP package.  We urge you to reconsider building the {\tt mpich}
package on your system!)
\item
Within {\tt SiteSpecific} set {\tt MPI\_LIBS} to point to the directory
containing the MPI/MPE libraries, and to specify the names of the link
archives, for example\\
\mbox{\tt MPI\_LIB=-L/usr/local/mpi/lib -lmpi -lmpe}.\\
Note that if you use the version of {\tt mpicc} which is distributed
with {\tt mpich} you may not need to have any of the MPI libraries
referenced here; the compiler may find them automatically.
\item
Within {\tt SiteSpecific} set {\tt MPI\_INCLUDES} to point to the
directory which contains the MPI and MPE header ({\tt *.h}) files, for
example\\
\mbox{\tt MPI\_INCLUDES = -I/usr/local/mpi/include}.
\item
Within {\tt SiteSpecific} set {\tt MPICC} to the name of your
local MPI C compiler, for example:\\
\mbox{\tt MPICC = /usr/local/bin/mpicc}.\\
You can include any compilation flags (say, {\tt -g}) on this line also.
\item
If you have the MESA or GL library installed set {\tt HAVE\_GL= true}, otherwise
set it to {\tt false}.  In this latter case, the routines making GL/MESA calls 
will not be compiled. 
\item
Within {\tt SiteSpecific} set {\tt GL\_LIBS} to point to the directory
containing the GL/MESA libraries, and to specify the names of the link
archives, for example\\
\mbox{\tt GL\_LIBS= -L/usr/local/lib -lMesaGLU -lMesaGL \$(XLIBS)}.\\
Note that the functions in the MESA/GL library make calls to the X library
and you will have to specify the location of the X libraries for example\\
\mbox{ XLIBS = -L/usr/X11/lib -L/usr/X11R6/lib -lX11 -lXext -lXmu -lXt -lXi -lSM -lICE
}.
\item
Within {\tt SiteSpecific} set {\tt GL\_I} to point to the
directory which contains the GL/MESA header ({\tt *.h}) files, for
example\\
\mbox{\tt GL\_I = -I/usr/local/include/GL}.
\item
If you intend to use CLAPACK, set {\tt WITH\_CLAPACK = true},
otherwise set it to false.  Within {\tt SiteSpecific} set {\tt
CLAPACK\_LIB} to point to the directory containing the CLAPACK
libraries and {\tt LCLAPACK} and {\tt LBLAS} to the
(platform-specific) names of the clapack and blas libraries
respectively excluding the leading `lib'. Further set {\tt F2C\_LIB}
to point to the directory containing the f2c libraries and {\tt
F2C\_INC} to point to the directory containing the f2c.h include file.
\item
If you intend to use the FRAME code, set {\tt BUILD\_FRAME = true},
otherwise set it to false.  In this latter case, any FRAME-specific defines
are ignored, and no code that makes use of FRAME function calls is
compiled.
\item
Within {\tt SiteSpecific} set {\tt FRAME\_DIR} to point to the directory
which contains the LIGO/VIRGO format FRAME software, for example\\
{\tt FRAME\_DIR=/usr/local/frame}.\\  
This directory should contain {\tt lib/libFrame.a} and {\tt include/FrameL.h}. 
If you don't need the FRAME
libraries, just leave this entry blank.
\item 
Within {\tt SiteSpecific}, if you want to use GRASP for real-time
analysis in the Caltech 40-meter lab, set {\tt EPICS\_INCLUDES} to point to the directory
containing the EPICS {\tt *.h} include files, and set {\tt EPICS\_LIBS} to point
to the directory containig the EPICS libraries.  Finally, you need to uncomment
the {\tt BUILD\_REALTIME} define statement.  If you do not intend to use
your GRASP installation for real-time analysis in the 40-meter lab, simply leave
these three definitions commented out with a hash sign ({\tt \#}).
\item
At the bottom of {\tt SiteSpecific} are several define statements,
which are currently commented out.  These are primarily intended for
production code; by undefining these lines you replace a cube root
function and some trig functions in the code with faster (but less
accurate) in-line approximations.  We suggest that you leave these
commented out.  (You might want to consider uncommenting them if you
are burning thousands of node hours on a large parallel machine - but
you do so at your own risk!)
\item
There are also lines that are currently commented out, which allow you
to overload functions defined in the libraries and reference libraries
of optimized functions.  Once again, leave these commented out unless you
want to replace standard {\it Numerical Recipes} functions with optimized
versions.  Currently, we support several sets of optimized libraries:
\begin{itemize}
\item
The CLASSPACK optimized FFT's for the Intel Paragon.
\item
The Sun Performance Library's optimized FFT for the Sun SPARC
architecture.  Note: believe it or not, this is {\it slower} than the
public domain equivalent.  We recommend that you use the FFTW package
instead!
\item
The Cray/SGI optimized FFT for the RS10000 and other MIPS architectures.
Note: believe it or not, this is {\it slower} than the
public domain equivalent.  We recommend that you use the FFTW package
instead!
\item
The DEC extended math library (DXML) optimized FFT for the DEC AXP
architecture.  [This is slightly faster, or slightly slower, than FFTW,
depending upon the array size.]
\item
The FFTW (Fastest Fourier Transform in the West), which will run on
any computer.  This is a public domain optimized FFT package, available
from the web site:\\
\htmladdnormallink{{\tt http://www.fftw.org/}.}{http://www.fftw.org/}
If you don't have an optimized FFT routine for your computer, we highly
recommend this -- it is a factor of three (or more) faster than {\it
Numerical Recipes}. We include glue routines for both FFTW version 1 and
version 2.  The latter is simpler to install and fractionally more efficient.
\item
The IBM Extended Scientific Subroutine Library (ESSL) optimized FFT routine. 
Note: this has only been tested on the IBM SP2 machine. 
\end{itemize}
Further details may be found in the {\tt src/optimization} subdirectory
of GRASP.  If you want to use these optimized library routines, first
go into the appropriate subdirectory of {\tt src/optimization} and build
the optimized library routine using the {\tt makefiles}'s that you find
there, then uncomment the appropriate lines in {\tt SiteSpecific} and
follow the instructions given here.
\item
To install the Gravitational Radiation Toolbox which provides an interface between
GRASP and Matlab, comment out the line {\tt BUILD\_GRTOOLBOX = false} and
uncomment {\tt BUILD\_GRTOOLBOX = true}. Then edit the variables
{\tt MEX}, {\tt MEXFLAGS}, and {\tt EXT} appropriately. You will then have add the
directories {\tt src/GRtoolbox} and {\tt src/examples/examples\_GRtoolbox} to your Matlab path.
\item
Now, in the top level GRASP directory, execute the shell script {\tt
InstallGRASP}, by typing the commands:\\
{\tt chmod +x InstallGRASP\\
./InstallGRASP SiteSpecific} (or {\tt SiteSpecific.mysite} if appropriate)\\
From here on, the remainder of the installation should proceed
automatically.  The {\tt InstallGRASP} script takes information contained
in the {\tt SiteSpecific} file (or in the file named in the first
argument of {\tt InstallGRASP} such as {\tt SiteSpecific.mysite})
and uses it to create {\tt Makefile}'s
in each {\tt src} subdirectory, and runs {\tt make} in each of those
directories.
\item
If you want to ``uninstall'' GRASP so that you can begin the
installation procedure again, cleanly, a script has been provided
for this purpose.  To execute it, type:\\
{\tt ./RemoveGRASP}\\
and wait until the script reports that it has finished.
Note: when you have sucessfully completed this process, please email
us
a copy of your {\tt SiteSpecific} file and we will put it into the
{\tt Examples\_SiteSpecific/ } directory of future GRASP releases.
\end{enumerate}
The {\tt Makefile} in each directory is
constructed by concatenating the file named in the first argument
of {\tt InstallGRASP} (typically {\tt SiteSpecific}) with a file
called {\tt Makefile.tail} in each individual directory.  If you want to
try changing the compilation procedure, you can modify the {\tt Makefile}
in a given directory.  However this will be created each time that you
run {\tt InstallGRASP}; for changes to become permanent they should
either be made in {\tt SiteSpecific} or in the {\tt Makefile.tail}'s.

Note that this installation procedure and code has been tested on the
following types of machines:  Sun 4 (Solaris), DEC AXP (OSF), IBM SP2
(AIX), HP 700 (HPUX), Intel (Linux), Intel Paragon.  
There is a problem on some SGI (IRIX) machines.  If you get error messages
reading:
\begin{verbatim}
   ...
   if: Expression Syntax.
   *** Error code 1 (bu21)
   GRASP did NOT complete installation successfully
\end{verbatim}
this can be fixed by setting the shell to {\tt bash} before running
InstallGRASP:
\begin{verbatim}
   SGI> setenv SHELL /usr/local/bin/bash
   SGI> ./InstallGRASP SiteSpecific
\end{verbatim}
If you run into
problems with our installation scripts, please let us know so that we
can fix them.

If you want to experiment with GRASP or to write code of your own, a
good way to start is to copy the {\tt Makefile} and the example ({\tt
*.c}) programs from the \mbox{\tt src/examples} directory into a directory
of your own.  You can then edit one of the example programs, and type
``{\tt make}" within your directory to compile a modified version of
the program.

If you wish to modify the code and libraries distributed with GRASP (in
other words, modify the functions described in this manual!) the best
idea is to use {\tt cp -r} to recursively copy the entire GRASP
directory structure (and all associated files) into a private directory
which you own.  You can then install your personal copy of GRASP, by
following the directions above.  This will permit you to modify source
code within any of the {\tt src} subdirectories; typing {\tt make}
within that directory will automatically re-build the GRASP libraries
that you are using.  By the way, if you are modifying these functions
to fix bugs or repair problems, or if you have a ``better way" of doing
something, please let us know so that we can consider incorporating
those changes in the general GRASP distribution.

\subsubsection{Stupid Pet Tricks}
There are a number of simple things that one can do during or after the
installation process that may make GRASP easier to maintain and/or use at your
site. For example, if is often extremely convenient for debugging
purposes to have a GRASP library ({\tt libgrasp.a}) constructed with all the
symbol table information turned on, and another library constructed
with all the optimization switches turned on.  Users who want their
code to run as fast as possible can link to the optimized library.
Users who want to track down problems within GRASP, or to step through
internal GRASP functions can link to the debug library.  You can accomplish this
easily by building two separate GRASP libraries, as follows.  (Note: since the
normal C-compiler debugging option is {\tt -g} the debug library has
a {\tt \_g} appended to its name.)
\begin{itemize}
\item
Edit your {\tt SiteSpecific.mysite} file so that the debugging
switches are turned on ({\tt CFLAGS = -g}, typically).  You may also
want to build the GRASP example programs with the ``debug'' versions
of the Numerical Recipes libraries, in which case you should set 
{\tt LRECIPES=recipes\_cg} in {\tt SiteSpecific.mysite}. In simiilar
fashion you might also choose to link to FRAME libraries compiled with debugging
turned on.
\item
Build GRASP as described above, by running\\
{\tt InstallGRASP SiteSpecific.mysite}.
\item
Make a copy of the GRASP library:\\
{\tt cp /usr/local/GRASP/lib/libgrasp.a
/usr/local/GRASP/lib/libgrasp\_g.a}
\item
Make a copy of the ``debug'' GRASP example programs:\\
{\tt cp -r /usr/local/GRASP/bin /usr/local/GRASP/bin\_g}\\
Note: the {\tt /usr/local/GRASP/bin} directory contains {\it links}
to the actual executables, but for most unix systems this copy command
will copy the actual files.  Your mileage may vary -- choose the copy
option
which copies the files {\it not} the links!
\item
{\it Remove} your GRASP installation (i.e. everything but the
library and the executables, and original GRASP package) by typing:\\
{\tt RemoveGRASP}
\item
Modify {\tt SiteSpecific.mysite} so that the optimization options are
turned on (typically, {\tt CFLAGS = -O}).  You should also set the
Numerical Recipes library to the optimized versions, typically via
{\tt LRECIPES=recipes\_c} in {\tt SiteSpecific.mysite}.  In simiilar
fashion you might also choose to link to FRAME libraries compiled with optimization
turned on.
\item
Install GRASP again:\\
{\tt InstallGRASP SiteSpecific.mysite}
\end{itemize}
Your GRASP installation will now contain {\it two} GRASP libraries:
{\tt /usr/local/GRASP/lib/libgrasp.a} and 
{\tt /usr/local/GRASP/lib/libgrasp\_g.a} and two sets of
executables, in 
\linebreak[4]
{\tt /usr/local/GRASP/bin} and 
{\tt /usr/local/GRASP/bin\_g}.

Another useful trick is if you are building versions of GRASP for
several different architectures, on a shared {\tt /usr/local/} disk.
Here the procedure is the following:
\begin{itemize}
\item
Create a {\tt SiteSpecific.arch1} file for the first machine
type.
\item
Install GRASP in the usual way:\\
{\tt InstallGRASP SiteSpecific.arch1}
\item
Copy the libraries:\\
{\tt cp /usr/local/GRASP/lib/libgrasp.a
/usr/local/GRASP/lib/libgrasp\_arch1.a}
\item
Make a copy of the GRASP example programs:\\
{\tt cp -r /usr/local/GRASP/bin /usr/local/GRASP/bin\_arch1}\\
Note: the {\tt /usr/local/GRASP/bin} directory contains {\it links}
to the actual executables, but for most unix systems this copy command
will copy the actual files.  Your mileage may vary -- choose the copy
option
which copies the files {\it not} the links!
\item
{\it Remove} your GRASP installation (i.e. everything but the
library and the executables, and original GRASP package) by typing:\\
{\tt RemoveGRASP}
\item
Return to the first step above, and begin this process again, but this
time for the second machine architecture (i.e. change {\tt
arch1} to {\tt arch2} above).
\end{itemize}
This method will avoid duplication of source files, documentation,
etc, while still providing a set of libraries and executables for
different machine types.

\subsection{Conventions used in this manual}
The conventions used in this manual are not strict ones.  However we do
observe a few general rules:
\begin{enumerate}
\item
Words or lines that you might type on a computer (commands, filenames,
names of C-language functions, and so are) are generally indicated
in {\tt teletype font}.
\item
When a function is described, the arguments which are {\it inputs} and those
which are {\it outputs} (or those which are both) are indicated.  Thus,
for example the (fictional!) addition function\\ \mbox{\tt add(int a, int b,int* c)} which sets
{\tt *c = a+b} is described by:
\begin{description}
\item{\tt a:} Input. One of the two integers that are added together.
\item{\tt b:} Input. The second of these integers.
\item{\tt c:} Output. Set to the sum of {\tt a} and {\tt b}.
\end{description}
Note that technically this is incorrect, because of course in C even
the ``output arguments" are really just inputs; they are pointers to an
address in memory that the routine is supposed to modify.  And
technically, the statement that ``c is set to..." is not correct, since
in fact it is the integer pointed to by c (denoted *c) that is set.
However we find that this convention makes it much easier to read the
function descriptions!
\item
Most of the time, the example programs using GRASP functions are given
explicitly in the manual, so you can see the GRASP functions ``in use".
Because these examples are illustrative, they are generally ``pared down"
as much as possible (for example, default values of adjustable parameters
are hard-wired in, rather than prompted for).
\item
Routines and example programs in GRASP generally begin with
the line:\\
{\tt \#include "grasp.h"}\\
which includes the prototypes for all GRASP functions as well as
the library header files {\tt stdio.h}, {\tt stdlib.h},
{\tt math.h}, {\tt values.h}, and {\tt time.h}.  The GRASP include
file {\tt "grasp.h"} can be found in the {\tt include} subdirectory
of GRASP.
\end{enumerate}

\subsection{How to add your contributions to future GRASP releases.}
As we have explained, the general idea of GRASP is to have a
collection
of documented and tested code available for use by the
gravitational-wave
detection community.  Many people have made significant contributions
to this package, and we would welcome any additional contributions.

In order to minimize the effort involved in making additions to GRASP,
and in order to ensure that they are properly included and available
to
all, here are some guidelines about how to contribute:

\begin{itemize}
\item
The contributions must be structured in such a way that they
    can be installed using the standard GRASP installation scripts, and
    they respect the GRASP file hierarchy.
\item
The contributions must be documented in {\bf TeX}, following
    the same general style as this manual (we try not 
    to be {\it too} nit-picky!).
\item
In general, be sure that you are using (and modifying!) the
current
release of GRASP.  This makes it much easier for me to merge your
additions in with the existing code.
The danger is that if you modify files from an old release of GRASP,
where other corrections/changes have subsequently been made, then I need
to try and merge these changes into what you have done.  It's easier for
everyone if you are modifying the most current files.
If you contact me in advance, I
can also give you some idea about how many changes have already been
made to the current release, and what the schedule is for the next release.
One way to find the absolute ``most current" version of a file is
to get it from the GRASP development source tree, which is at\\
\htmladdnormallink{{\tt http://www.lsc-group.phys.uwm.edu/$\sim$ballen/grasp-distribution/GRASP/}}
{http://www.lsc-group.phys.uwm.edu/~ballen/grasp-distribution/GRASP/}.\\
Before sending me
a revised file to incorporate in GRASP, please {\tt diff} it with the corresponding file in this
directory, to make sure that the only differences are ones that you have deliberately made!
\end{itemize}

If you want to make "small" changes to GRASP, for example to modify a function to add
extra functionality, to repair something that is broken, or to add some additional
functions in one section, then please do the following:
\begin{enumerate}
\item
Provide documentation in the form of a modified file:
    {\tt doc/man\_*.tex}.  I will merge your changes into the general GRASP
    distribution.  For clarity, let's assume that you have added a
    utility function, and have modified {\tt doc/man\_utility.tex} by adding
    a description of your function(s) to it.  Remember, {\it software can never
    be better than its documentation}.
\item
In any modifications of the documentation {\tt *.tex} files, please be sure to
{\it spell-check} the files before you send them to me.  Use either the {\tt spell}
or {\tt ispell} utilities, or some other alternative.
\item
 You should provide additional lines to add to {\tt doc/make\_tex\_from\_C}.
    This is a script that automatically converts any {\tt *.c} example files
    that you would like to include in the manual into {\tt .tex} files.  In
    general, you should have an example program which shows a very
    simple use of your function.
\item
You should provide any figures which have been included in your
    "modified" {\tt doc/man\_*.tex} file in postscript:
    {\tt doc/Figures/MYPACKAGE1.ps}, {\tt doc/Figures/MYPACKAGE2.ps}, and so on.
Note that the postscript files produced by many plotting packages are 
excessively large, often several MB.   For figures produced from
programs like this, it is more efficient to use a bitmap to describe the entire image. 
Full details of how to produce such bitmaps may be found at\\
\htmladdnormallink{{\tt
    http://xxx.lanl.gov/help/bitmap}}{http://xxx.lanl.gov/help/bitmap}.
The following extract describes `the easiest way to do bitmapping'
using XV: 

``First display the original figure on the screen somehow (e.g. with
ghostview). Then use the `Grab' button in XV to snatch a copy of the
displayed image into XV's buffer (after selecting `Grab' you can do
this either by clicking the left mouse button on the desired window
(which grabs the whole window), or by holding down the middle mouse
button and dragging (which selects a region)).

Once the image is in XV's buffer, you can manipulate it. You should
use `Autocrop' or `Crop' to remove any excess margins around the
figure. Then save it (as gif, jpeg, color postscript or greyscale
postscript). If resaving as postscript you must click the XV
`compress' box for extra compression.'' 
\item
You should provide a modified version of (for example) {\tt src/utility/utility.c}
    (this modified source contains your additions, merged into the
    standard GRASP release) and additional example programs
    demonstrating your functions in\\
    {\tt src/examples/examples\_utility/example1.c}, \\
    {\tt src/examples/examples\_utility/example2.c}, and so on.
\item
You should provide a modified version of {\tt include/grasp.h} (or the
additional lines to merge into this header file).  This header should contain
proto-types for any functions which you have added to GRASP, which you would
like to make publicly-available. In general, please try to avoid putting
``documentation" in this header file: it should go into the manual and into
the source files.
\end{enumerate}
If you are doing this (modifying or extending existing GRASP
functions) please
do a search of the existing GRASP source code to verify that
your changes do not break existing code.  Or, if necessary, make
modifications to the existing code, and send me those modified files
as well as the materials above.

On the other hand, you might have grander plans!  You might not want
to make ``small'' changes - you
might want to include a major new section in GRASP, for modelling
another type of source, or for a type of analysis which is different
than anything currently in the package.  Let's assume that you
want to provide a major new GRASP package or facility called
MYPACKAGE.  In this case:
\begin{enumerate}
\item
You should provide documentation in the form of a file:
    {\tt doc/man\_MYPACKAGE.te}x, which has the same format as (for example)
    {\tt doc/man\_inspiral.tex}.  I will modify {\tt doc/manual.tex}
    by putting a line:\\
    {\tt include\{man\_MYPACKAGE\}}\\
    into {\tt doc/manual.tex}, to include your
    contribution to the manual. Reference should be in the same format
    as the existing ones, and will be added to the {\it references} section of
    {\tt manual.tex}.  Make sure that you modify {\tt doc/man\_intro.tex} to describe any
    additional directories that you have added to the {\tt src} tree.  Remember, {\it
    software can never
    be better than its documentation}.

\item
In any modifications of or additions to the documentation {\tt *.tex} files, please be sure to
{\it spell-check} the files before you send them to me.  Use either the {\tt spell}
or {\tt ispell} utilities, or some other alternative.
\item
You should provide additional lines to add to {\tt doc/make\_tex\_from\_C}.
    This is a script that automatically converts any {\tt *.c} example files
    that you would like to include in the manual into {\tt .tex} files.
\item
You should provide figures in postscript form with names derived from
    your package name if that is possible.  For example: {\tt
    doc/Figures/MYPACKAGE1.ps},
    {\tt doc/Figures/MYPACKAGE2.ps}.  These figures should be included in your
    {\tt doc/man\_MYPACKAGE.tex} file.
\item
You should provide source code in {\tt src/MYPACKAGE/MYPACKAGE.c} and
    example programs in\\
    {\tt src/examples/examples\_MYPACKAGE/example1.c},\\
    {\tt src/examples/examples\_MYPACKAGE/example2.c}, and so on.  The code
    in \linebreak[4] {\tt src/MYPACKAGE/MYPACKAGE.c} contains the actual functions that
    you have provided.  Any executable example programs that use these
    functions should be in the {\tt src/examples} path.
\item
You should provide a modified header file {\tt include/grasp.h} or
additional lines to merge into this, declaring prototypes for any
publicly-available functions.
\item
You should provide ``tail" parts of the Makefiles:\\
    {\tt src/examples/examples\_MYPACKAGE/Makefile.tail}, and \\
    {\tt src/MYPACKAGE/Makefile.tail}.  You can see\\
    {\tt src/examples/examples\_inspiral/Makefile.tail} for an example.
    Please follow the syntax of this fairly closely: the structure is there for good
    reasons.  If you provide a file that works on your own system it may not
    work on other people's systems -- but if you follow our style it probably will.
\item
Be sure to modify the {\tt InstallGRASP} utility to include a ``build"
in any directories that you have added to the {\tt src} tree.
\end{enumerate}
In general, the ``rule of thumb'' here is that you should try not to
add functions which substantially overlap existing GRASP functions.
Either modify the existing GRASP functions (as described below) to add
the extra functionality or use them ``as is''.

I would be grateful for a bit of advanced warning about any additions to
GRASP (but a uuencoded gzipped tarfile or shar file dropped in my mailbox
{\it will} get my rapid attention, if it contains these different items,
because that makes it {\it easy} for me to incorporate it!)  The best
format for this file is to make it contain only the files that you are
adding to GRASP, or those files from the most current GRASP distribution
that are being modified, {\it with exactly the correct directory tree
structure}.  This makes it easy for me to unpack your contributions and
merge them into the general GRASP distribution.
One way to find the absolute ``most current" version of a file is
to get it from the GRASP development source tree, which can be reached from\\
\htmladdnormallink{{\tt http://www.lsc-group.phys.uwm.edu/$\sim$ballen/grasp-distribution/GRASP/}}
{http://www.lsc-group.phys.uwm.edu/~ballen/grasp-distribution/GRASP/}.  Before sending me
a revised file to incorporate in GRASP, please {\tt diff} it with the file in this
directory, to make sure that the only differences are ones that you have deliberately made!

Note: it is a good idea to check your code in the following ways:
\begin{itemize}
\item Error messages in your code should not be done with
{\tt fprintf(stderr,$\cdots$)}, but should be dealt with
by calling the GRASP error handler.  Look at any of the current
GRASP code to see how this is done, or read Section \ref{ss:errors}
on the error handler.
\item Make sure that it passes by {\tt lint} cleanly.
\item Stick to POSIX operating systems calls, only.  Remember that
your code needs to run on {\it other} platforms.  The fact that the
code works correctly on your platform does not guarantee similar
behavior on other types of machines.  This item, and the previous one,
will go a long way towards ensuring that.
\item If you are using the {\tt gcc} compiler, make sure that
the {\tt -Wall} option (which enables all warnings) does not issue any
warnings.
\item If you have access to more than one type of machine, please test
your code on both a 32-bit and 64-bit machine if possible.
\item If you have access to both big-endian and little-endian
machines, please test you code on both of those also.
\end{itemize}
Please remember that many people will be running {\it your} code on
platforms which are different than yours.  The fact that your code
runs properly on your platform does not mean that it will run properly
on other ones as well.  The best way to ensure this is to eliminate
the types of problematic constructions that {\tt lint} and {\tt -Wall}
warn about, and to test your code on a couple of different machines.

One final ({\bf important}) note.  When your code has been sucessfully
integrated into the GRASP package, we will issue a new release of
GRASP as quickly as possible.  As soon as this release is available,
we {\it strongly} recommend that you {\it throw away} your ``personal"
working copies of the files that you have been creating or modifying,
install this latest release of GRASP, and make {\it new} copies of
the various files to work from.  The reason is this: in the
process of including your material into GRASP we have probably made
a number of changes to it.  If you {\it don't} follow our suggestion,
make additional changes to your own files and send them to us, we will
not be very receptive (as you are forcing us to perform the unpleasant
task of merging your changes with our earlier ones).

\subsection{How to use the GRASP library from ROOT.}

ROOT is an interactive public-domain environment for data analysis developed
at CERN.  Details about it can be found at
\htmladdnormallink{{\tt http://root.cern.ch/}}
{http://root.cern.ch/}.
Within the root environment, you can make use of the GRASP library.
To do this, however, you need to produce a shared-object version of the
GRASP library, and then load it into ROOT.  The following instructions
on how to do this were contributed by Damir Buskulic ({\tt buskulic@lapp.in2p3.fr}),
and have been tested under Linux.
\begin{enumerate}
\item
In building {\tt libgrasp.a} and {\tt librecipes\_c.a} be sure that you have
a {\it position independent code} option.  For the {\tt gcc} compiler this
is {\tt -fPIC}.  This is needed to build a proper shared library.
\item
The ROOT environment variables should have been set during install,
especially {\tt \$ROOTSYS}.
\item
Issue the following three commands:
\begin{itemize}
\item
To build the interface functions for the ROOT C interpreter:\\
{\tt cint -w1 -zlibgrasp -nG\_\_cpp\_grasp.cc -D\_\_MAKECINT\_\_
-DG\_\_MAKECINT\\
-c-1 -DG\_\_REGEXP -DG\_\_SHAREDLIB -DG\_\_OSFDLL
-D\_\_cplusplus \\ -I\$ROOTSYS/include -I<path\_to\_GRASP>/include/grasp.h}\\

This will create files {\tt G\_\_cpp\_grasp.cc} and {\tt G\_\_cpp\_grasp.h}
which need to be compiled.
\item
Compile these files with:\\
{\tt gcc -O -fPIC -DG\_\_REGEXP -DG\_\_SHAREDLIB -DG\_\_OSFDLL\\
-I\$ROOTSYS/include -c G\_\_cpp\_grasp.cc}\\
\item
Build the shared library with the command:\\
{\tt gcc -shared -o libgrasp.so ./G\_\_c\_grasp.o
<path-to-libgrasp.a>/libgrasp.a <path-to-numrecipes>/librecipes\_c.a}\\
creating a {\tt libgrasp.so} file.  You can put this file anywhere that you wish.
\end{itemize}
Note: it is important to put the files and libraries in the order given
above.
\item
Launch ROOT and use the command {\tt gSystem->Load("libgrasp");} to have the routines
from the GRASP library available within ROOT.
\end{enumerate}
Note that {\tt cint} has some trouble with prototypes for variable-length-argument (varargs) functions. 
These types of functions are used to implement the GRASP error handler \ref{ss:errors}.
For this reason
the GRASP include file {\tt include/grasp.h} has some lines which are
automatically {\it not} included if the file is being read by {\tt cint}.  Thus the
GRASP error-handling functions can't be called from within ROOT (but you
wouldn't want to do this anyway).

Note that one can compile the Frame library for use with ROOT in exactly the
same way.  You replace {\tt grasp.h} by {\tt FrameL.h} and {\tt libgrasp.a} by {\tt libFrame.a}
(always compiled with {\tt -fPIC}). There is no need for {\tt librecipes} in the
Framelib case, because it does not make use of {\it Numerical Recipes}.
\clearpage



