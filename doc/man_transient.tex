% GRASP: Copyright 1997,1998,1999  Bruce Allen
% $Id: man_transient.tex,v 1.5 1999/07/11 21:22:18 ballen Exp $
\section{GRASP Routines: Supernovae and other transient sources}
\label{s:transient}
\setcounter{equation}0

In any classification scheme for natural phenomena, the boundaries between
classifications run the risk of becoming broad and blurry. While previous
sections have avoided this by considering very narrowly defined classes of
gravitational wave sources, this section considers a much broader class of
sources, and as such is a kind of ``catch all'' for a variety of sources. 
These sources, however, all share in common two properties which unite them in
terms of practical detection strategies:
\begin{enumerate}
\item They produce gravitational radiation in the Ligo frequency and
sensitivity band for a short time. The definition of short time is, of course,
relative. However, if the time scale over which a source is visible to LIGO is
of the order of days, as opposed to seconds, then it will likely be more
suited to the analysis tools for quasi-periodic and periodic sources in
section \ref{s:periodic}. 
\item Astrophysical models of these sources are somewhat crude.
Those short lived sources from which the gravitational radiation can be
modelled accurately enough to permit matched filtering, may be dealt 
with using the tools of earlier sections (cf. section~\ref{s:inspiral}).
\end{enumerate}
At the time of this writing, the sources which have these properties
are thought to be primarily related to supernova events.
\newpage

\subsection{Centrifugal Hang-up of Core Collapse Supernovae}
\label{ss:LS}

Massive stars which have burnt their nuclear fuel will collapse under the pull
of their gravitational self attraction, sometimes triggering a supernova event. 
This collapse can lead to situations in which significant gravitational 
radiation is produced.  In particular, there are various scenarios in which
the collapsing core is thought to ``hang-up'' in a non-axisymmetric 
configuration and radiate this assymmetry away through gravitational waves.

A promising scenario of this type was explored by Lai and Shapiro \cite{LS95}
and proceeds as follows:a
stellar core with some initial angular momentum collapses. As the collapse
proceeds, the ratio $\beta=T/W$ of the rotational energy ($T$) to the
gravitational potential energy ($W$) will vary inversely with the core 
radius. If $\beta$ becomes large enough, the evolution of non-axisymmetric 
modes of the core become unstable. Such instabilities are well known, being 
first described by Chandrasekhar \cite{Chandra69}. There are two 
possibilities. If $\beta$ lies in a critical band of values between 
$\sim 0.17$ and $\sim 0.27$, the non-axisymmetric bar mode ($\ell=m=2$) of the 
core will be {\em secularly} unstable and grow due to either radiation 
reaction or viscosity. This long lasting instability is expected to produce 
significant gravitational radiation. If $\beta$ exceeds the upper limit of 
this band ($\beta > 0.27$) then non-axisymmetric modes become {\em dynamically} 
unstable. The transition through dynamical instability is rapid and results in 
a nearly axisymmetric configuration with $\beta < 0.27$ but still much 
larger than $0.17$. The core therefore again enters the regime of secular 
instability. In either scenario the onset of the secular instability occurs 
at approximately neutron star densities. If $\beta < 0.17$ for the entire 
collapse, then no significant gravitational radiation is expected from this
mechanism.

Lai and Shapiro have calculated the waveform of the gravitational
radiation emitted by a secularly unstable core based on crude Newtonian fluid 
ellipsoids models without viscosity. They find \cite{LaiPC}:
\begin{equation}
   \left. \begin{array}{l}
   h_+(t) = A(f_{max},r,M,d)~u(t)^{2.1}\sqrt{1-u(t)}\,(1+\cos^2i)\cos\Phi(t),\\
   h_\times(t) = 2\, A(f_{max},r,M,d)~u(t)^{2.1}\sqrt{1-u(t)}\,\cos i~\sin \Phi(t).
   \end{array} \right\}
   \label{eq:stress}
\end{equation}
where:
\begin{description}
\item{$f_{max}$} is the initial frequency of the gravitational wave (typically
$\stackrel{<}{\sim}$ 800 Hz).
\item{$r$} is the radius of the core (typically $\sim$ 10 km).
\item{$M$} is the mass of the core (typically $\sim$ 1.4 $M_\odot$).
\item{$d$} is the distance from the core (source).
\item{$i$} inclination angle (see section \ref{s:inspiral}).
\item{$A(f_{max},r,M,d)$} is the amplitude function
\begin{equation}
   A=\frac{1}{2} \frac{M^2}{rd}\left\{\begin{array}{lr}
      \left(\frac{\overline{f_{max}}}{1756}\right)^{2.7},&
         \overline{f_{max}}\le 415 {\rm Hz},\\
			\\
      \left(\frac{\overline{f_{max}}}{1525}\right)^{3.0},&
         \overline{f_{max}} > 415 {\rm Hz}.
   \end{array} \right.
   \label{eq:amp}
\end{equation}
\item{$\overline{f_{max}}$} = $f_{max} \sqrt{r_{10}^3 / M_{1.4}}$ where $r_{10} =
r/10$km and $M_{1.4}=M/(1.4M_\odot)$.
\item{$u(t)$}=$f(t)/f_{max}$ is given by the frequency evolution equation
\begin{equation}
   \frac{du}{dt} = \left(\frac{M}{r}\right)^{5/2} \left(\frac{A}{0.16}\right)^2
   \sqrt{\frac{M_{1.4}}{r_{10}}} ~~u^{5.2} (u-1).
   \label{eq:LSfreq}
\end{equation}
\item{$\Phi(t)$} is the phase, given implicitly by
\begin{equation}
   f(t)=\frac{1}{2\pi}\frac{d\Phi}{dt}(t). \label{eq:LSphase}
\end{equation}
\end{description}
This model is implemented below.
\newpage

\subsection{Structure: {\tt LS\_physical\_constants}} \label{ss:LSstruct}
The physical parameters describing the hung-up core are passed in a
structure {\tt struct}\\ {\tt LS\_physical\_constants}. The fields are:\\
{\tt struct LS\_physical\_constants \{}
\begin{description}
\item{\tt float mass}: The mass of the stellar core in solar masses (typically
$\sim 1.4 M_\odot$).
\item{\tt float radius}: The radius of the stellar core in km (typically $\sim
10$km).
\item{\tt float distance}: The distance from the stellar core to the detector 
in Megaparsecs.
\item{\tt float fmax}: The maximum frequency of gravitational wave emitted by
the stellar core (typically $\stackrel{<}{\sim}$ 800 Hz).
\item{\tt inclination angle}: angle between spin axis of stellar core and line
of sight to detector in radians.
\item{\tt Phi\_0}: the initial phase of the gravitational wave. 
\end{description}
{\tt \}}
\newpage

\subsection{Function: {\tt LS\_freq\_deriv()}}
\label{ss:Lfd}
{\tt void LS\_freq\_deriv(float t, float u[], float dudt[])}
This function gives the derivative of frequency (actually, $u$ which
is the frequency divided by the maximum frequency) as a function of 
time, (see (\ref{eq:LSfreq})). The amplitude $\left(\frac{M}{r}\right)^{5/2} 
\left(\frac{A}{0.16}\right)^2 \sqrt{\frac{M_{1.4}}{r_{10}}}$ must be declared
and assigned a value as an external variable called {\tt A}.

The arguments are:
\begin{description}
\item{\tt t}: Input. The time at which the derivative is to be taken.
\item{\tt u[]}: Input. An array of initial values (in this case, containing
exactly one element) of the dependent variables (in this case, only $u$).
\item{\tt dudt[]}: Output. An array of derivatives (in this case, containing
exactly one element) of derivates of the dependent variables (in this case,
$\partial u /\partial t$).
\end{description}

\begin{description}
\item{Authors:} Warren G. Anderson, warren@ricci.phys.uwm.edu 
\item{Comments:} This function is required by numerical recipes odeint. See
the numerical recipes manual for more information.
\end{description}
\newpage

\subsection{Function: {\tt LS\_phas\_and\_freq()}}
{\tt void LS\_phas\_and\_freq(double Phi[], float u[], float A, 
           float fmax, float dt, int n\_samples)}\\
This function integrates Lai and Shapiro's frequency and phase evolution 
equations, (\ref{eq:LSfreq}) and (\ref{eq:LSphase}), for a hung-up 
collapsed core. We use numerical recipes odeint, an adaptive step size 
4th order Runge-Kutta integrator for the frequency integration and a
simple trapezoidal integration for the phase.

The arguments are:
\begin{description}
\item{\tt Phi[]}: Output. An array which holds the phase of the gravitational
wave in radians at equally spaced time intervals. {\tt Phi[]} must be allocated
sufficient memory before being passed to {\tt LS\_phas\_and\_freq()}.
\item{\tt u[]}: Output. An array which holds the reduced frequency (frequency
divided by $f_{max}$) of the gravitational wave at equally spaced time
intervals. {\tt u[]} must be allocated sufficient memory before being passed
to {\tt LS\_phas\_and\_freq()}.
\item{\tt A}: Input. The Amplitude $A$ as calculated in (\ref{eq:amp})
\item{\tt fmax}: Input. The maximum frequency, $f_{max}$. Usually taken from
the appropriate field of a\\ {\tt LS\_physical\_constants} structure.
\item{\tt dt}: Input. The time interval (in seconds) at which the phase and 
frequency values should be output.
\item{\tt n\_samples}: Input. The number of phase and frequency values to be
output (i.e. the number of elements in the arrays {\tt Phi[]} and {\tt u[]}).
\end{description}

\begin{description}
\item{Authors:} Warren G. Anderson, warren@ricci.phys.uwm.edu and Patrick Brady,
patrick@tapir.caltech.edu
\end{description}
\newpage

\subsection{Function: {\tt LS\_waveform()}}
{\tt void LS\_waveform(float h[], struct LS\_physical\_constants phys\_const,
           float sky\_theta, float sky\_phi, float polarization, float dt, 
               int n\_samples) }\\
This function calculates the stress as measured by a LIGO like detector
due to a hung-up collapsing core. It uses the routines {\tt 
LS\_phas\_and\_freq} to obtain the wave phase and reduced frequency at
regular time intervals, calculates $h_+$ and $h_\times$ at each interval according 
to (\ref{eq:stress}), and then converts these into a single detector stress 
at each interval using beam pattern factors calculated with 
{\tt beam\_pattern}. Note that the external variable {\tt A} is assigned a
value here for use in {\tt LS\_freq\_deriv} (see \ref{ss:Lfd}).

The arguments are:
\begin{description}
\item{\tt h[]}: Output. An array which holds the detector stress due the 
gravitational wave of the hung-up core at equally spaced time intervals. 
{\tt h[]} must be allocated sufficient memory before being passed to {\tt
LS\_waveform()}.
\item{\tt phys\_const}: Input. A structure of type {\tt
LS\_physical\_constants} (see subsection \ref{ss:LSstruct}) which contains the
physical parameters of the hung-up core.
\item{\tt sky\_theta}: Input. The polar angle from zenith in radians.
\item{\tt sky\_phi}: Input. The azimuthal angle (measured counter clockwise
from first arm) in radians.
\item{\tt polarization}: Input. The polarization angle in radians.
\item{\tt dt}: Input. The time interval (in seconds) at which the detector
stress values should be output.
\item{\tt n\_samples}: Input. The number of detector stress values to be 
output (i.e. the number of elements in the array {\tt h[]}).
\end{description}

\begin{description}
\item{Authors:} Warren G. Anderson, warren@ricci.phys.uwm.edu and
Patrick Brady,
patrick@tapir.caltech.edu
\item{Comments:} This function currently calculates the stress at a user
defined distance from the hung-up core. For realistic distances, this leads
to small values (of the order of 10e-22 at 10 MPc). If one wishes to have
numbers of the order of unity, simply set the distance in the structure
{\tt phys\_const} to be 10e-22.
\end{description}
\newpage

\subsection{Example: {\tt LS\_filter} program}
This program uses {\tt LS\_waveform} to generate a gravitational wave 
detector stress waveform from a hung-up collapsing core and print 
it to the screen {\tt (stdout)}. This is the same thing that one would
generate as a filter if one wanted to use matched filtering for such waves.
Since this waveform is not expected to be an accurate representation of an
astrophysical gravitational wave, due to the crude model by which it is
generated, it is probably better to think of this as simply an illustration of
the use of the {\tt LS\_waveform} function.

\lgrindfile{Includes/LS_filter.tex}

\begin{description}
\item{Author:} Warren G. Anderson, warren@ricci.phys.uwm.edu
\end{description}


